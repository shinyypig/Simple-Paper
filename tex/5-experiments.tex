\section{Experiments}

In this section, three experiments are carried out, including one with synthetic data, and two with real data. To demonstrate the superiority of our method MinMGM, we also implement six other popular MGM methods for comparison, including four one-shot methods, and two iterative methods. For convenience, the compared methods are denoted MatchSync \cite{pachauri2013solving}, MatchLift \cite{chen2014nearoptimal}, MatchALS \cite{zhou2015multiimage}, MatchMCF \cite{wang2018multiimage}, JOMGM \cite{yan2013joint}, and CDMGM \cite{7001592}, respectively. Among them, MatchSync, MatchLift, MatchALS and MatchMCF are one-shot methods, while JOMGM and CDMGM are iterative methods. Besides, we adopt the codes from \cite{wang2018multiimage} for the four one-shot methods, and the two iterative methods are coded by us. 

The pairwise affinity matrices of the graphs to be matched are constructed by following the previous work \cite{cho2010reweighted}, where all possible edges are connected. The similarity between two nodes or edges is given by the Gaussian kernel $\exp(-||\bm f_i - \bm f_j|| / \sigma)$, where $\sigma$ is the normalization factor, $\bm f_i$ and $\bm f_j$ are the feature vectors of the two nodes or edges, respectively.

In addition to the pairwise affinity matrices, MatchMCF also requires the coordinates of the nodes and the number of inliers. Therefore, in the synthetic experiment, the generated graphs are embedded on a 2D plane. For the sake of fairness, the number of inliers is set as that of the nodes in MatchMCF. Moreover, the famous Reweighted Random Walks for Graph Matching \cite{cho2010reweighted}, which is also known as RRWM, is used as the two-graph matching method for all methods.

\subsection{Synthetic Dataset}

In this experiment, we randomly generate a bunch of nodes on a 2D plane. Then, different zero-mean Gaussian noise is added to the coordinates of the nodes to generate a series of graphs to be matched. Besides, we also evaluate the anti-outlier performance of all methods by randomly adding outliers to the graphs to be matched. The length of the edges are used as the features when generating the affinity matrices, and $\sigma$ is set as $0.2\max(||\bm f_i - \bm f_j||)$. To obtain a statistical result, for each setting, all methods are implemented for 200 times. In the anti-noise experiment (\cref{fig:5.1}), the number of nodes is set to 10, and the standard deviation of the noise ranges from 0 to 0.2. Meanwhile, in \cref{fig:5.2}, the number of inliers is set to 5, and the number of outliers ranges from 0 to 5.

\begin{figure}[htb!]
    \centering
    \begin{subfigure}{.23\textwidth}
        \centering
        \includegraphics[width=\textwidth]{./img/acc_4_noise.pdf}
        \caption{}
        \label{fig:5.1.1}
    \end{subfigure}
    \begin{subfigure}{.23\textwidth}
        \centering
        \includegraphics[width=\textwidth]{./img/acc_6_noise.pdf}
        \caption{}
        \label{fig:5.1.2}
    \end{subfigure}
    \begin{subfigure}{.23\textwidth}
        \centering
        \includegraphics[width=\textwidth]{./img/acc_8_noise.pdf}
        \caption{}
        \label{fig:5.1.3}
    \end{subfigure}
    \begin{subfigure}{.23\textwidth}
        \centering
        \includegraphics[width=\textwidth]{./img/acc_10_noise.pdf}
        \caption{}
        \label{fig:5.1.4}
    \end{subfigure}
    \caption{The anti-noise performance. (a) 4 graphs. (b) 6 graphs. (c) 8 graphs. (d) 10 graphs.}
    \label{fig:5.1}
\end{figure}

\begin{figure}[htb!]
    \centering
    \begin{subfigure}{.23\textwidth}
        \centering
        \includegraphics[width=\textwidth]{./img/acc_4_outlier.pdf}
        \caption{}
        \label{fig:5.2.1}
    \end{subfigure}
    \begin{subfigure}{.23\textwidth}
        \centering
        \includegraphics[width=\textwidth]{./img/acc_6_outlier.pdf}
        \caption{}
        \label{fig:5.2.2}
    \end{subfigure}
    \begin{subfigure}{.23\textwidth}
        \centering
        \includegraphics[width=\textwidth]{./img/acc_8_outlier.pdf}
        \caption{}
        \label{fig:5.2.3}
    \end{subfigure}
    \begin{subfigure}{.23\textwidth}
        \centering
        \includegraphics[width=\textwidth]{./img/acc_10_outlier.pdf}
        \caption{}
        \label{fig:5.2.4}
    \end{subfigure}
    \caption{The anti-outlier performance. (a) 4 graphs. (b) 6 graphs. (c) 8 graphs. (d) 10 graphs.}
    \label{fig:5.2}
\end{figure}

As can be seen in \cref{fig:5.1}, the performance of all methods generally deteriorates as the standard deviation of the noise and the number of graphs increase. Due to the fact that the one-shot methods share the same input, their accuracy curves are close to each other. CDMGM adopts a strategy that ensures the objective function will not decrease during iteration, therefore, its performance is obviously better than that of JOMGM. Notice that maximizing the minimum pairwise similarity also ensures that the sum of all pairwise similarities is maximized as much as possible. Meanwhile, due to the efficiency of the approximation algorithm used in our method, MinMGM always obtains the highest matching accuracy as shown in \cref{fig:5.1}.

It can be found in \cref{fig:5.2} that when there exists more outliers in all graphs to be matched, the matching accuracy of all methods inevitably decreases. At the same time, when the number of graphs increases, the performance of all compared methods except JOMGM remains relatively constant in this experiment. Further, it can be found that since the number of inliers is set as that of the nodes, MatchMCF can not effectively identify the outliers. Furthermore, for a set of matched nodes, if an outlier is involved, the corresponding minimum pairwise similarity will decrease significantly. Therefore, MinMGM has the ability to automatically identify the outliers, especially when there are more graphs to be matched. From \cref{fig:5.2.4}, it can be told that when there are 10 graphs to be matched, the matching accuracy of MinMGM is significantly higher than that of compared methods.

\begin{figure}[htb!]
    \centering
    \begin{subfigure}{.23\textwidth}
        \centering
        \includegraphics[width=\textwidth]{./img/acc_graph1.pdf}
        \caption{}
        \label{fig:5.3.1}
    \end{subfigure}
    \begin{subfigure}{.23\textwidth}
        \centering
        \includegraphics[width=\textwidth]{./img/acc_graph2.pdf}
        \caption{}
        \label{fig:5.3.2}
    \end{subfigure}
    \caption{Accuracy versus the number of graphs. (a) Noise with standard deviation of 0.15, no outlier. (b) 3 outliers, no noise.}
    \label{fig:5.3}
\end{figure}

Additionally, the curves of accuracy versus the number of graphs are also plotted in \cref{fig:5.3}, where the number of graphs ranges from 3 to 10. It can be found in \cref{fig:5.3.1} that when there are more graphs in the anti-noise experiment, MinMGM generally obtains a better performance than that of other compared methods. From \cref{fig:5.3.2}, it can be seen intuitively that as there are more graphs to be matched, MinMGM gradually achieves a matching accuracy close to 1, which is obviously superior to the compared MGM methods.

\subsection{CMU House/Hotel Dataset}

The CMU House Dataset contains a sequence of 111 images of a toy house taken from different viewpoints, and the CMU Hotel Dataset contains a sequence of 101 images. For each image, there exist 30 hand-labeled landmark points, which are consistent for the same object.  In this experiment, we randomly select 4 images with different sequence gaps from the same dataset, and for each set of 4 images, 10 inliers are randomly selected from the labeled points for evaluation. Besides, experiments with different number of outliers and graphs to be matched are carried out, so that the performance of all methods can be comprehensively compared. In this section, the graphs are constructed by following the same way as that in the synthetic data experiment. The average matching accuracy curves of 500 runs are plotted in \cref{fig:5.5}.

\begin{figure}[htb!]
    \centering
    \begin{subfigure}{.23\textwidth}
        \centering
        \includegraphics[width=\textwidth]{./img/cmu_house_3.pdf}
        \caption{}
        \label{fig:5.5.1}
    \end{subfigure}
    \begin{subfigure}{.23\textwidth}
        \centering
        \includegraphics[width=\textwidth]{./img/cmu_hotel_3.pdf}
        \caption{}
        \label{fig:5.6.1}
    \end{subfigure}
    \begin{subfigure}{.23\textwidth}
        \centering
        \includegraphics[width=\textwidth]{./img/cmu_house2_10.pdf}
        \caption{}
        \label{fig:5.5.2}
    \end{subfigure}
    \begin{subfigure}{.23\textwidth}
        \centering
        \includegraphics[width=\textwidth]{./img/cmu_hotel2_10.pdf}
        \caption{}
        \label{fig:5.6.2}
    \end{subfigure}
    \begin{subfigure}{.23\textwidth}
        \centering
        \includegraphics[width=\textwidth]{./img/cmu_house3_10.pdf}
        \caption{}
        \label{fig:5.5.3}
    \end{subfigure}
    \begin{subfigure}{.23\textwidth}
        \centering
        \includegraphics[width=\textwidth]{./img/cmu_hotel3_10.pdf}
        \caption{}
        \label{fig:5.6.3}
    \end{subfigure}
    \caption{The results for CMU house/hotel dataset. (a) (c) (e) CMU house dataset. (b) (d) (f) CMU hotel dataset. (a) (b) 3 outliers, 4 graphs, and sequence gap ranges from 1 to 19. (c) (d) 4 graphs, sequence gap of 10, and number of outliers ranges from 0 to 6. (e) (f) 3 outliers, sequence gap of 10, and number of graphs ranges from 3 to 10.}
    \label{fig:5.5}
\end{figure}

% \begin{figure}[htb!]
%     \centering
%     \begin{subfigure}{.23\textwidth}
%         \centering
%         \includegraphics[width=\textwidth]{./img/cmu_hotel_3.pdf}
%         \caption{}
%         \label{fig:5.6.1}
%     \end{subfigure}
%     \begin{subfigure}{.23\textwidth}
%         \centering
%         \includegraphics[width=\textwidth]{./img/cmu_hotel2_10.pdf}
%         \caption{}
%         \label{fig:5.6.2}
%     \end{subfigure}
%     \begin{subfigure}{.23\textwidth}
%         \centering
%         \includegraphics[width=\textwidth]{./img/cmu_hotel3_10.pdf}
%         \caption{}
%         \label{fig:5.6.3}
%     \end{subfigure}
%     \caption{The results for CMU hotel dataset. (a) 3 outliers, 4 graphs, and sequence gap ranges from 1 to 19. (b) 4 graphs, sequence gap of 10, and number of outliers ranges from 0 to 6. (c) 3 outliers, sequence gap of 10, and number of graphs ranges from 3 to 10.}
%     \label{fig:5.6}
% \end{figure}

% \begin{figure}[htb!]
%     \centering
%     \begin{subfigure}{.23\textwidth}
%         \centering
%         \includegraphics[width=\textwidth]{./img/cmu_house_2.pdf}
%         \caption{}
%         \label{fig:5.6.1}
%     \end{subfigure}
%     \begin{subfigure}{.23\textwidth}
%         \centering
%         \includegraphics[width=\textwidth]{./img/cmu_house_5.pdf}
%         \caption{}
%         \label{fig:5.6.2}
%     \end{subfigure}
%     \begin{subfigure}{.23\textwidth}
%         \centering
%         \includegraphics[width=\textwidth]{./img/cmu_hotel_2.pdf}
%         \caption{}
%         \label{fig:5.6.3}
%     \end{subfigure}
%     \begin{subfigure}{.23\textwidth}
%         \centering
%         \includegraphics[width=\textwidth]{./img/cmu_hotel_5.pdf}
%         \caption{}
%         \label{fig:5.6.4}
%     \end{subfigure}
%     \caption{CMU house/hotel dataset experiment with fixed number of outliers. (a) CMU house dataset with 2 outliers. (b) CMU house dataset with 5 outliers. (c) CMU hotel dataset with 2 outliers. (d) CMU hotel dataset with 5 outliers.}
%     \label{fig:5.6}
% \end{figure}

\begin{table*}[htbp!]
    \centering
    \small
    \begin{tabular}{cccccc}
    \hline
    Accuracy  & Car            & Duck           & Face       & Motorbike      & Winebottle     \\ \hline
    MatchSync & 0.826 & 0.458 & \textbf{1} & 0.957 & 0.843 \\ 
    MatchLift & 0.836 & 0.462 & \textbf{1} & 0.960 & 0.844 \\ 
    MatchALS  & 0.840 & 0.456 & \textbf{1} & 0.957 & 0.843 \\ 
    MatchMCF  & 0.842 & 0.461 & \textbf{1} & 0.961 & 0.845 \\ 
    JOMGM     & 0.850 & 0.464 & \textbf{1} & 0.973 & 0.820 \\ 
    CDMGM     & 0.905 & 0.501 & \textbf{1} & \textbf{0.992} & 0.848 \\ 
    MinMGM    & \textbf{0.909} & \textbf{0.539} & \textbf{1} & 0.987 & \textbf{0.854}\\\hline
    \end{tabular}
    \caption{The matching accuracy of all classes.}
    \label{tab:5.1}
\end{table*}

It can be seen from \cref{fig:5.5.1,fig:5.6.1} that, when the sequence gap becomes larges, the performance of all methods generally decreases. This is because that the larger the sequence gap is, the more the graphs to be matched are different. Similarly, in \cref{fig:5.5.2,fig:5.6.2}, it can be seen that the number of outliers also has a great impact on the matching accuracy of all methods. As for the number of graphs, it can be found in \cref{fig:5.5.3,fig:5.6.3} that due to the accumulation of matching errors, the performance of all compared methods decreases as the number of graphs increases. On the contrary, the matching accuracy of our method is relatively stable when matching different number of graphs. Moreover, MinMGM obtains the highest matching accuracy in all cases, which further demonstrates the superiority of MinMGM in the anti-outlier performance.

\subsection{Willow Object Dataset}

The willow object dataset \cite{willow} contains 304 images from categories, collected from Caltech-256 \cite{caltech} (face, duck and winebottle) and Pascal VOC 2007 \cite{voc2007} (car and motorbike). For each image, there exists an annotation file with 10 distinctive points, which are consistent for the same class.

In this experiment, due to the fact that the size of the objects varies in different images, we use the angles between the edges and the horizontal line as the features. When construct the affinity matrices, the normalization factor $\sigma$ is set as 0.3 for all object classes. In \cref{tab:5.1}, for each trail, we randomly select 10 images from the same class, while in \cref{fig:5.4}, the number of graphs ranges from 3 to 10. Both the results shown in \cref{tab:5.1,fig:5.4} are obtained by averaging the results of 1000 trails.

From \cref{tab:5.1}, it can be found that though there is no outlier, MinMGM still can achieve the highest mean matching accuracy for most classes. It can be seen in \cref{fig:5.4}, due the deformation between the objects in different images, when the number of graphs increases, the performance of all compared methods gradually decreases. However, MinMGM can still achieve a better performance than that of other compared methods. This further demonstrates that by maximizing the minimum pairwise similarity, the sum of all similarities is also forced to be as large as possible. Additionally, when there are numerous graphs to be matched, MinMGM generally shows a more obvious advantage over other compared methods. This indicates that the proposed approximation algorithm can effectively obtain a potentially better solution.

\begin{figure}[htb!]
    \centering
    \includegraphics[width=.25\textwidth]{./img/acc_willow_all.pdf}
    \caption{The average matching accuracy of all classes.}
    \label{fig:5.4}
\end{figure}
