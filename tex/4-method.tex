\section{引用}

SimplePaper 使用 cleveref 宏包来实现引用的自动编号。至于文献引用,则使用了 biblatex 宏包,相关配置已在 simplepaper.cls 文件中定义。

\subsection{图表引用}
因为使用了 cleveref 宏包,所以在引用图表时,可以直接使用 $\backslash$cref$\{$fig:example$\}$ 这样的命令,而无需标注这是图还是表。比如,只需要输入如下简单的代码,就可以生成对应格式的引用(需要注意的是,逗号后面不要加空格,否则会出错):
\begin{lstlisting}
\cref{eq:1} | \cref{fig:1} | \cref{tab:1} | \cref{fig:1,tab:1} | \cref{fig:1,tab:1,thm:1} | \cref{fig:cameraman2_1,fig:cameraman2_2,fig:cameraman2_3}
\end{lstlisting}
引用的结果如下:
\begin{shaded}
\cref{eq:1} | \cref{fig:1} | \cref{tab:1} | \cref{fig:1,tab:1} | \cref{fig:1,tab:1,thm:1} | \cref{fig:cameraman2_1,fig:cameraman2_2,fig:cameraman2_3}
\end{shaded}

\subsection{文献引用}

将需要引用的文献的 bib 格式放在 ref.bib 文件中,然后在文档中使用 $\backslash \text{cite}\{\text{key1,key2,key3}\}$ 的方式引用即可。比如,下面的代码:
\begin{lstlisting}
这是一个引用\cite{simplepaper}。
\end{lstlisting}
\begin{shaded}
这是一个引用\cite{simplepaper}。
\end{shaded}
然后在$\backslash\text{end}\{\text{document}\}$之前插入如下代码即可输出参考文献:
\begin{lstlisting}
\printbibliography[heading=simplepaper]
\end{lstlisting}